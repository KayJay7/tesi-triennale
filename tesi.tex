\documentclass[12pt, a4paper]{article}
\usepackage{authblk}
\usepackage{listings}
\usepackage{xcolor}
\usepackage{graphicx}
\usepackage{float}
\usepackage[a-2b,mathxmp]{pdfx}[2018/12/22]
\usepackage[pdfa]{hyperref}
\usepackage{indentfirst}
\usepackage{pdfpages}
\usepackage{siunitx}
\usepackage{colorprofiles}
\usepackage{amsmath}
\usepackage{amsfonts}


% stile per i blocchi di codice
% non so se serviranno
\definecolor{codegreen}{rgb}{0,0.6,0}
\definecolor{codegray}{rgb}{0.5,0.5,0.5}
\definecolor{codepurple}{rgb}{0.58,0,0.82}
\definecolor{backcolour}{rgb}{0.95,0.95,0.92}

\lstdefinestyle{mystyle}{
    backgroundcolor=\color{backcolour},
    commentstyle=\color{codegreen},
    keywordstyle=\color{magenta},
    numberstyle=\tiny\color{codegray},
    stringstyle=\color{codepurple},
    basicstyle=\ttfamily\footnotesize,
    breakatwhitespace=false,         
    breaklines=true,                 
    captionpos=b,                    
    keepspaces=true,                 
    numbers=left,                    
    numbersep=5pt,                  
    showspaces=false,                
    showstringspaces=false,
    showtabs=false,                  
    tabsize=2
}

\lstset{style=mystyle}
% fine stile

\newcommand{\michele}[1]{{\color{red}Michele:~}{\itshape``{#1}''}}

\begin{titlepage}
    \title{Analisi di scalabilità dell'implementazione in Golang del calcolo AbU}
    \author{Bruniera Alvise}
    \affil{Università degli studi di Udine}
    \date{\today}
\end{titlepage}

\begin{document}
\maketitle

\newpage
\begin{abstract}
    `GoAbU' è un'implementazione del calcolo `AbU' (Attribute-based memory Updates) per programmare sistemi distribuiti di piccoli dispositivi IoT. In questa tesi ci poniamo l'obbiettivo di studiarne il comportamento all'aumentare del numero di nodi, sotto diversi tipi di carico, utilizzando il simulatore `abusim'. Quindi stabilire un metodo ragionevole e realizzare degli strumenti, per analizzare il comportamento di `goabu'.\\
    A questo scopo abbiamo realizzato alcuni programmi e script. In particolare `aeg' per generare parametricamente configurazioni per il simulatore, in modo da creare più facilmente configurazioni complesse con numero variabile di nodi, anche centinaia. Ed una serie di script per misurare il traffico di rete di ogni nodo (in termini di quantità di byte e pacchetti) ed il traffico riportato a livello di transazioni da uno dei nodi (quantità di byte e di transazioni), e per estrarre grafici e metriche utili da questi dati.\\
    Abbiamo analizzato il traffico per 10, 50 e 100 nodi, sotto tre tipi di carico: nessun carico (idle), carico alto ma realistico (medio), carico molto alto (alto). In tutti i casi, con solo 10 nodi il carico è accettabile.\\
    Il traffico è risultato molto alto sia sotto carico medio che alto, in particolare, sotto carico alto, diventava problematico già da 50 dispositivi. Tuttavia, confrontando i risultati con il traffico riportato a livello di libreria da `goabu', risulta che la compressione svolta dal middleware riesce a rendere il traffico molto minore di quanto potrebbe essere.

    % mai buttare via niente ;)
    %  In idle, il numero di dispositivi non sembra influire molto sul traffico della rete, sia il numero di pacchetti che di byte resta molto basso, anzi diminuisce la dimensione media dei pacchetti. Sotto carico medio, il numero di nodi influenza in modo quadratico il carico della rete, inoltre, 50 dispositivi sono sufficienti a generare circa 2 GB di traffico, mentre 100 si avvicinano ai 10 GB, posiamo aspettarci di vedere questo tipo di traffico in applicazioni reali. Sotto carico alto, anche se con 10 nodi il traffico è accettabile (300 MB), è comunque 5 volte più alto del carico medio. Con più nodi invece diventa insostenibile già a 50 nodi, paragonabile al traffico del carico medio con 100 nodi; quando si arriva a 100 nodi supera i 15 GB di traffico ed uno dei benchmark è arrivato fino a 25. \michele{Non andrei così in dettaglio nell'abstract, metterei solo un riassunto di cosa vogliamo testare (senza numeri)} \\
    % , con rapporto di compressione tra il 2,2 ed il 2,6
\end{abstract}

\newpage
\tableofcontents

\newpage
\section{Il calcolo AbU}

AbU~\cite{MP-ICTAC-2021} (Attribute-based memory Updates) è un paradigma di calcolo distribuito basato sull'utilizzo di regole ECA (event-condition-action) per sincronizzare e scambiare informazioni tra grandi quantità di nodi.\\
Le regole non sono molto diverse da delle normali ECA: ogni regola è legata ad una variabile del nodo, e quando questa variabile viene modificata (da una regola o da un dispositivo di input), se una condizione specificata è verificata, esegue l'azione.
La caratteristica che distingue AbU è la possibilità di specificare regole ``esterne'', che agiscano sulle variabili degli altri nodi del sistema. Una regola esterna può specificare condizioni sia su variabili locali che esterne, mentre nelle azioni possono specificare solo variabili esterne. Un'azione avrà effetti su un nodo solo se possiede tutte le variabili esterne menzionate nella regola.
Utilizzando queste regole, la memoria dei nodi viene aggiornata in base ad i suoi attributi (le variabili che possiede il nodo) da cui il nome.\\
In questo modo, è possibile programmare un sistema distribuito completamente disaccoppiato nello spazio, poiché l'effetto di un'azione sugli altri nodi dipende da loro attributi e non richiede di conoscere la posizione delle variabili.
Una regola interna AbU è della forma \lstinline{rule <name> on <variable> for <condition> do <action>}, ad esempio \lstinline{rule step0_0 on a0_0 for a0_0 > 0 do a0_1 = a0_0; a0_0 = 0}. Una regola esterna, invece, è della forma \lstinline{rule <name> on <variable> for all <condition> do <action>} (notare \lstinline{for all} al posto di \lstinline{for}) e può accedere a variabili remote (non per l'attivazione), ad esempio \lstinline{rule activate0 on a0_4 for all this.a0_4 > 0 && (ext.id == 8) do ext.a0_0 = (this.a0_4 - 1)}.

\href{https://github.com/abu-lang/goabu}{GoAbU} è un'implementazione di questo modello di calcolo scritta in linguaggio go.
È costruita utilizzando la libreria \href{https://github.com/hashicorp/memberlist}{memberlist} come middleware per la comunicazione in gruppo e la scoperta dei nodi, ``Grule'' \href{https://github.com/hyperjumptech/grule-rule-engine} come rule engine, ed il framework \href{https://github.com/hybridgroup/gobot/}{gobot} per renderla utilizzabile in ambito IoT.\\
Il focus principale della tesi sarà sia definire un metodo di test ragionevole e riproducibile per testare un'implementazione del calcolo AbU (e gli strumenti per eseguire test), che un'analisi di GoAbU stesso.

\paragraph{Limitazioni del calcolo AbU}

Durante lo sviluppo dei test abbiamo incontrato due limitazioni del linguaggio, una delle quali facilmente risolvibile una volta definita la semantica di una funzionalità aggiuntiva, l'altra invece potrebbe eventualmente essere mitigata ma non risolta del tutto.\\
Il linguaggio non ammette che le regole esterne eseguano azioni sulle variabili locali. Questo è in parte perché se le regole impostassero una variabile locale al valore di una esterna, il valore finale della variabile locale non solo dipenderebbe dagli agenti esterni, ma anche dall'ordine in uci eseguono le azioni. Tuttavia, questo problema non sussiste se nel right-hand-side dell'assegnamento compaiono solo costanti e variabili locali. Tuttavia, permettere questa funzionalità (che aiuterebbe nella scrittura dei programmi) richiede di rivedere la semantica delle azioni esterne, per decidere in che momento va eseguita l'azione.\\
% supercazzola che probabilmente bisognerà ridurre
L'altro problema invece riguarda la consistenza del sistema. Per il CAP theorem~\cite{brewer2000towards}, sappiamo che un sistema distribuito non può essere contemporaneamente ``consistente'', ``resistente alla partizione'' e ``disponibile''; AbU è sia resistente che disponibile, quindi non si può ottenere consistenza, ed anche la consistenza eventuale sarebbe fuori dagli obbiettivi del paradigma.
% veramente il problema di cui voglio parlare
Tuttavia è particolarmente suscettibile al problema dei lost update. Nello specifico, se ad esempio un nodo \lstinline{foo} possiede la variabile intera \lstinline{x = 0} (e nessun altro nodo la possiede) $n$ nodi causano (con successo) l'esecuzione di una azione esterna del tipo \lstinline{ext.x = ext.x + 1;}, il valore di \lstinline{x} al termine dell'esecuzione delle azioni, il valore di \lstinline{x} sarà contenuto in $[0,n]$, ma l'effettivo valore dipenderà interamente dal momento in cui le $n$ azioni vengono inserite nella coda degli eventi di \lstinline{foo}.
Come annunciato in precedenza non si può rendere AbU completamente consistente, quindi già a prescindere non è mai garantito che tutti gli update abbiano effetto. Tuttavia nella situazione attuale, anche quando la rete funziona correttamente, non è possibile implementare un contatore ``affidabile'' che venga incrementato da più nodi.\\
Entrambi questi problemi non rientrano nell'obbiettivo della tesi, ma sono venuti alla luce perché per la prima volta il calcolo AbU è stato usato per qualcosa al di fuori di uno showcase. Quindi era il caso di esporli.

\section{Setting sperimentale}

\subsection{Il simulatore AbUsim}

I test sono stati eseguiti utilizzando il ``simulatore'' \href{https://github.com/abu-lang/abusim}{AbUsim}. Si basa sull'utilizzo di docker ed è progettato sulla falsa riga di docker compose, ovvero preso in input un file yml che descrive una serie di container, li avvia e mette in comunicazione.
I file di configurazione di `AbUsim' permettono di specificare un elenco di dispositivi che eseguiranno `goAbU'. Ogni dispositivo ha un suo elenco di variabili (che possono essere inizializzate arbitrariamente) ed un elenco di regole in linguaggio `AbU'.
Per comodità è possibile specificare un prototipo in ogni dispositivo. I prototipi specificano anch'essi un elenco di regole e di variabili che i dispositivi erediteranno. In questo modo si possono raggruppare regole in comune tra tanti dispositivi.
Si possono configurare anche il tick time (il tempo che il dispositivo aspetterà tra le esecuzioni delle regole) e l'immagine docker da usare per i nodi ed il coordinatore.

Il parsing delle regole di ogni nodo viene fatto direttamente dal simulatore, che dopo aver costruito l'elenco di regole e variabili di ogni nodo, codificherà gli alberi di sintassi in \lstinline{json} e successivamente in \lstinline{base64} per passarli come argomenti al programma in esecuzione su ogni nodo, questa informazione sarà utile più avanti. È possibile collegare altri nodi esterni alla rete del simulatore che eseguano goAbU, ma non è necessario per questa tesi.\\
Viene avviato anche un coordinatore, che gestisce la comunicazione con l'esterno sia per inviare comandi che per ricevere log. Questo è strettamente parte del simulatore e non un elemento che ci si aspetta tri trovare in una reale applicazione di AbU, che invece è pensato per essere decentralizzato. Va notato, allo scopo del progetto, che la rete virtuale su cui viaggiano i messaggi di goAbU è separata da quella su cui viaggiano i messaggi di gestione ed i log.\\
Opzionalmente si può avviare un container che offre un'interfaccia web per controllare i nodi, quindi permette di verificarne lo stato, inviare comandi, ed impostare il livello dei log. Tutto ciò può essere fatto anche programmaticamente tramite una libreria python.

\subsection{Problemi secondari}

Durante lo sviluppo del generatore di test abbiamo incontrato alcune limitazioni od errori del simulatore. Questi errori sono indipendenti da goAbU, e riguardano puramente il funzionamento del simulatore, quindi verranno trattati solo brevemente.

\paragraph{Limite alla dimensione della configurazione}

Come già spiegato nella sezione precedente, la configurazione di un nodo (quindi l'elenco delle variabili e delle regole) viene codificato in un'unica lunga stringa, e questa viene passata al nodo come argomento del comando shell. Poiché la dimensione del vettore \lstinline{argv} è limitata (la dimensione effettiva dipende dal sistema operativo, ma solitamente si parla di $\qty{128}{\kibi\byte}$) anche la quantità e la lunghezza delle regole.
$\qty{128}{\kibi\byte}$ potrebbero sembrare molti, ma anche se fossero passate in plaintext significherebbe che (supponendo le regole siano lunghe tendenzialmente $128$ carattere) si possono avere solo $1024$ regole, quando l'obbiettivo sarebbe quello di ammettere diverse migliaia di regole. In realtà la codifica attualmente utilizzata per la configurazione, è meno efficiente e limita ulteriormente la quantità di regole, in modo dipendente dalla complessità sintattica di ogni regola.
Una soluzione proposta e testata in un fork del simulatore prevede di non passare la configurazione come argomento, ma attraverso il buffer di input, che invece non è limitato. Questa soluzione funziona e può essere implementata semplicemente, ma non è l'unica possibilità.

\paragraph{Loop nel logger}

Una delle difficoltà incontrate non dipende dal simulatore, ma ha permesso di scoprire un problema del coordinatore (che tra le altre cose si occupa di raccogliere i log dai nodi).
La dimensione massima per le tabelle ARP impostata di default su molti sistemi operativi non è sufficiente a permettere a più di $\sim40$ nodi di comunicare, per poter testare più dispositivi bisogna impostare un limite più alto dal sistema operativo. Tuttavia, per come funziona il simulatore, quando un nodo non riesce a collegarsi agli altri si chiude dopo un certo tempo.
Ed in questo modo abbiamo scoperto che se uno nodo smette di rispondere alla richiesta di nuovi log, un thread del coordinatore rimane bloccato in un loop, utilizzando al $\qty{100}{\percent}$ una delle cpu.
Si tratta di un problema marginale poiché si verifica solo quando uno dei nodi termina inaspettatamente, cosa che non dovrebbe succedere se non per problemi della setting come questo limite alle tabelle ARP.

\paragraph{Ricompilazione delle regole}\label{ricompilazione}

L'ultimo problema, invece, riguarda una funzionalità mancante della libreria goAbU ma necessaria per l'applicazione nel mondo reale di AbU.
Un rule engine è efficiente a trovare le regole da eseguire perché utilizza una struttura dati complessa, ma per costruirla deve prima eseguire una computazione complessa. E Grule non fa eccezione, costruire la struttura dati dal set di regole è costoso, sembra richiedere costo $\Theta(x^2)$, quindi avviando il simulatore su una grande quantità di regole (non necessario per questa tesi) potrebbe passare molto tempo prima che i nodi siano completamente avviati.
Ovviamente per il simulatore non si può fare molto, ma per un dispositivo reale che viene programmato una volta sola con un migliaio di regole, non è accettabile aspettare diversi minuti ad ogni riavvio perché sta ricostruendo la struttura. Soprattutto considerato che un dispositivo del genere potrebbe essere molto più lento di un computer che esegue AbUsim, e quindi potrebbe diventare troppo lento ancora prima.
Per risolvere questo problema bisognerebbe aggiungere una funzionalità per memorizzare la struttura precompilata in modo che successivamente il dispositivo debba solo leggerla per tornare operativo dopo un riavvio. Per il simulatore questo non vale poiché lo scopo di AbUsim è la prototipazione, ed è previsto che ad ogni avvio la configurazione cambi, rendendo meno utile conservare la struttura.

\section{Generazione dei test}

\subsection{Il generatore aeg}

Uno degli strumenti principali utilizzati per l'analisi è il generatore di test \href{https://github.com/KayJay7/abusim-example-generator}{aeg} (AbUsim Example Generator). Volevamo un modo per poter generare configurazioni che stressassero una caratteristica a piacere di goAbU. L'utilizzo di questo strumento ha permesso sia di provare agilmente grandi configurazioni, ad esempio quelle da migliaia di regole per dispositivo che hanno permesso di scoprire i lunghi tempi di compilazione (\ref{ricompilazione}); che di apportare rapidamente correzioni alle configurazioni selezionate come test cases, e di poter generare facilmente configurazioni uguali che variassero solo nel numero di nodi.

\paragraph{Descrizione}\label{aeg:descrizione}

Le configurazioni generate con aeg seguono tutte una stessa struttura, che permette di avviare una computazione che coinvolgerà tutti i nodi e terminerà sempre. Tutti gli $a$ nodi di una configurazione sono simili tra loro: anno le stesse variabili ed eseguono regole uguali, che differiscono al massimo per alcuni valori hardcoded. Ogni nodo ha due variabili booleane \lstinline{start} e \lstinline{start_all} inizializzate a \lstinline{false} che sono usate per avviare una computazione, più $b\times c$ (altro su $b$ e $c$ più avanti) variabili intere con nome \lstinline{ai_j} con $i\in [0,b)$ e $j\in [0,c)$ (e.g.: \lstinline{a0_0}), tutte inizializzate a \lstinline{0}. Queste variabili rappresentano $b$ ``catene'', ciascuna lunga $c$ variabili. Ogni nodo ha anche una variabile intera \lstinline{id}, ciascuna inizializzata ad un valore diverso e mai modificata, sono utilizzate per indirizzare il singolo nodo.

Ogni dispositivo ha 3 ``regole di avvio'', una si attiva quando \lstinline{start_all} viene impostato a \lstinline{true}, e non fa altro che impostare la variabile \lstinline{start} a \lstinline{true} su tutti i nodi. Una resetta \lstinline{start_all}. Ed infine quando \lstinline{start} viene impostato a \lstinline{true}, una ``variabile iniziale'' (la variabile \lstinline{ai_0}) delle prime $d\leq b$ catene viene impostata a $f$, e la variabile \lstinline{start} viene resettata.
Dopodiché, alle prime $c-1$ variabili di ogni catena (quindi da $0$ $c-2$) corrisponde una regola che si attiva quando la sua variabile viene impostata ad un valore maggiore di $0$, e contemporaneamente copierà il valore della variabile in quella successiva nella catena (e.g.: da \lstinline{a1_2} ad \lstinline{a1_3}) e resetterà la sua variabile.
Quando la $c$-esima variabile dell'$i$-esima catena viene impostata ad un valore \lstinline{k>0}, non c'è una corrispondente regola ``normale''. Invece ce ne sono una locale che resetta il valore della variabile a $0$, ed una esterna che setterà a \lstinline{k-1} il valore della variabile iniziale dell'$i$-esima dei successivi $e$ dispositivi (il dispositivo \lstinline{id} attiva quelli da \lstinline{(id+1)%a} ad \lstinline{(id+e)%a}). In questo modo la parte di computazione sul dispositivo corrente è completata, e continua sui dispositivi successivi.
Dopo $f$ livelli, il valore che viene passato sulle catene è arrivato ad \lstinline{1}, e ai prossimi dispositivi viene passato il valore \lstinline{0}, che le regole del $f+1$-esimo livello di dispositivi ignoreranno.

\paragraph{Parametri}

Quindi il comportamento della configurazione è deciso dai (già elencati nella descrizione \ref{aeg:descrizione}) parametri:
\begin{itemize}
    \item $a$ (alt.: devices-number) Il numero di nodi nella configurazione
    \item $b$ (alt.: chains-number) Il numero di catene in ogni nodo (non tutte saranno attivate automaticamente)
    \item $c$ (alt.: chain-length) La lunghezza di ogni catena
    \item $d$ (alt.: chain-width) Il numero di catene attivate automaticamente con \lstinline{start=true}
    \item $e$ (alt.: devices-width) Il numero di dispositivi attivati al termine di una catena
    \item $f$ (alt.: devices-length) Il numero di livelli attraversati prima di arrivare a \lstinline{0}
\end{itemize}

Oltre

% Non sono sicuro che basti come dimostrazione, spero di sì perché non è stata facile da scrivere
\subsection{Terminazione forte (?)}

Per semplicità affronteremo la terminazione in due parti. La prima parte riguarda solo la parte ``interna'' della computazione, quella che si occupa di trasportare i valori nelle catene di variabili, ignorando i valori che potrebbero essere aggiunti da una regola esterna avviata da un altro nodo\label{terminazione:premesse:interna}.
E la seconda riguardante solo i dispositivi che vengono attivati dalle regole esterne, assumendo che la parte interna termini correttamente, attivando i nodi successivi\label{terminazione:premesse:esterna}.
Sarebbe possibile fornire una dimostrazione unica che tratti direttamente entrambe le parti della computazione, ma è molto più difficile.

\paragraph{Convergenza interna}\label{terminazione:interna}

La terminazione sarà dimostrata prima in maniera meno formale, considerando un sistema in cui la regola da eseguire viene scelta dando la precedenza quelle che scrivono le variabili più vicine alla fine di una catena, e dando la precedenza alle catene con indice più alto. Questo permette di fissare l'ordine di esecuzione delle regole in uno su cui è più comodo ragionare. E infine sarà accennato perché un ordine così stringente non è necessario.
Inoltre assumeremo, come annunciato ad inizio sezione~\ref{terminazione:premesse:interna}, che nessun nodo esterno imposti il valore di alcuna variabile, tuttavia in seguito spiegheremo perché non sia un problema nel paragrafo sulla convergenza totale~\ref{terminazione:totale}.

Sia $(a_{0,0},...,a_{0,c-1},a_{1,0},...,a_{1,c-1},...,a_{b-1,c-1})\in \mathbb{N}^{bc}$ il vettore che rappresenta lo stato interno di un generico nodo della configurazione (ogni variabile $a_{i,j}$ rappresenta la variabile \lstinline{ai_j}~\ref{aeg:descrizione}). È facile verificare che ogni regola normale viene attivata da uno stato del tipo $(...,a_{i,j},a_{i,j+1},...)$ e causa uno stato del tipo $(...,0,a_{i,j},...)$ (notare che l'indice della catena è sempre $i$) e causerà l'inserimento di una nuova regola nella coda degli eventi.
Ricordiamo che con l'ordinamento stabilito ad inizio paragrafo, la regola attivata dallo stato $(...,a_{i,j},a_{i,j+1},...)$ verrà eseguita solo quando $a_{i,j+1}=0$, altrimenti ci sarebbe una regola con precedenza più alta.
Una regola finale, invece è causata da uno stato del tipo $(...,a_{i,j},a_{i+1,0},...)$ e ne causa uno del tipo $(...,0,a_{i+1,0},...)$ senza causare nessun inserimento.

Sapendo che: la computazione termina se e solo se il vettore che rappresenta lo stato interno è $(0,...,0)$, altrimenti ci sarebbero ancora regole in coda, attivate quando sono state impostate le variabili diverse da \lstinline{0}; e che ogni regola causa uno stato di ordine lessicografico minore rispetto a quello che lo ha attivato; possiamo concludere che la computazione deve terminare per forza altrimenti potrebbe scendere di ordine lessicografico all'infinito.
Ed al termine di questa computazione, saranno state eseguite le regole esterne per attivare i nodi successivi, perché le uniche regole che non aggiungono nuove regole sono attivate dallo stesso stato che causa l'attivazione delle regole esterne.

AbU, per sua natura, non fa garanzie sull'ordine di esecuzione delle regole causate dallo stesso stato, quindi per completare la dimostrazione formale dobbiamo mostrare che funzionerebbe anche senza questo ordinamento.
Possiamo vedere che non importa perché: esiste una corrispondenza diretta tra le variabili e le regole, e non ci sono ``interferenze'' tra le regole causate dallo stesso stato (non cercano di leggere la stessa variabile o di scrivere la stessa), ed è sufficiente entrare una volta in uno stato ed uscirne subito per attivare una regola.
Quindi l'ordine non importa, e quindi la computazione interna termina sempre.\label{terminazione:interna:ordinamento}

\paragraph{Convergenza esterna}

Come annunciato all'inizio della sezione~\ref{terminazione:premesse:esterna}, affronteremo la convergenza esterna considerando corretta la parte interna della computazione. Ovvero assumendo che nodo che viene ``attivato'' da un valore \lstinline{k}, eventualmente attiverà $e$ nodi con un valore \lstinline{k-1}. Questo segue dalla dimostrazione precedente~\ref{terminazione:interna}.

Consideriamo l'attivazione di un nodo come il passaggio di un token, ogni token ha un'``intensità'' $k$, ovvero il numero di livelli che può attraversare. Quando un nodo elabora un token, questo viene spezzato in $e$ token di intensità $k-1$, e quando un token ha intensità $0$, il nodo che lo riceve lo scarta.
Nelle nostre computazioni, l'intensità del token corrisponde al valore \lstinline{k} con cui il nodo viene attivato, ovvero il valore che la regola finale esterna inserisce all'inizio della catena di variabili dei nodi successivi. Quando viene impostata \lstinline{start = true} su di un nodo, questo genera $b$ token di intensità $f$. Quando viene impostata \lstinline{start_all = true} su un qualsiasi nodo, tutti gli $a$ nodi generano $b$ token ciascuno di intensità $f$.

Sia $(t_f,t_{f-1},...,t_1)\in\mathbb{N}^f$ il vettore dei token, dove ogni variabile $t_k$ rappresenta la quantità di token di intensità $k$ in circolazione sulla rete. Notiamo che i token di intensità $0$ non vengono contati, perché sono scartati subito.
Possiamo osservare che ogni volta che un token di intensità $k$ viene consumato, il vettore passa da $(...,t_k,t_{k-1},...)$ a $(...,t_k-1,t_{k-1}+e,...)$. Quindi ad ogni consumo, il nuovo vettore ha ordine lessicografico più basso di quello iniziale.

Sapendo che la computazione esterna si ferma se e solo se non ci sono più token in circolazione, quindi il vettore ha valore $(0,...,0)$, e che ad ogni avanzamento della computazione l'ordine lessicografico scende, sappiamo che la computazione deve terminare per forza.

Ammettendo che più token vengano consumati contemporaneamente, il risultato non cambia, perché si passerebbe comunque ad uno vettore più in basso nell'ordine lessicografico.

\paragraph{Convergenza totale}\label{terminazione:totale}

Per mettere insieme i due risultati manca solo dimostrare che ammettere attivazioni esterne non impedisce la convergenza interna. Riuscendo a dimostrare questo, avremmo che internamente la computazione termina in ogni caso, ed esternamente termina se quella interna termina.

Dimostrare questa non interferenza è facile utilizzando lo stesso ordine rigido che abbiamo usato per rendere più breve la dimostrazione della convergenza interna. Con quell'ordine rigido, le regole esterne che attiverebbero il nodo catena non verrebbero mai selezionate per prime per l'esecuzione, perché scrivono sulle variabili più lontane dalla fine della catena.
In questo modo, il nodo comincerebbe ad eseguire le regole di attivazione solo dopo aver terminato la computazione interna già in corso. Quindi l'attivazione non interferirebbe con la convergenza.

Ovviamente AbU non prevede questo ordinamento, ma come spiegato nel paragrafo sulla convergenza interna~\ref{terminazione:interna:ordinamento}, imporre questo ordinamento non influisce sul comportamento della nostra computazione.

\subsection{Scelta delle configurazioni}

Qua parlo di cosa volevo ottenere dalle configurazioni e come ho scelto i parametri.

\section{Raccolta dati}

\subsection{Cosa }

\subsection{Ispezione dei nodi}

Qua parlo di come raccoglie i dati lo script, e quali informazioni ottengo dai log.

\subsection{Metriche}

Elenco delle metriche.

Parlo più nel dettaglio solo delle metriche che ci interessano di più.

\section{Risultati}

Qua metto solo i grafici sulle metriche più importanti di cui ho parlato prima.

\ref{appendix:grafici}

\subsection{Osservazioni}

Spiego le informazioni ottenute dai grafici, quindi espando un po' di quello che è già nell'abstract \\

\michele{Gli altri grafici li metterei come appendice, da inserire dopo la bibliografia}



\bibliographystyle{plain}
\bibliography{biblio}

\appendix

\section{Altri grafici}\label{appendix:grafici}

Il resto dei grafici

\end{document}